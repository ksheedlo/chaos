\documentclass[12pt, letterpaper]{article}
\title{Problem Set 3: Fractals}
\author{Ken Sheedlo}
\usepackage[pdftex]{graphicx}
\usepackage{fullpage}
\usepackage{verbatim}
\usepackage{amsfonts}
\usepackage{caption}
\usepackage{subcaption}

\begin{document}
\maketitle{}

\section*{Newton's Method Fractal}

I went to the Google Labs site and explored the fractal for Newton's Method on
$x^3 - 1 = 0$. Figure 1 shows a plot of the fractal zoomed in several levels 
deep.

\begin{center}
\includegraphics[scale=0.4]{google_newton_fractal.png}
\\
Figure 1: Fractal structure generated by Newton's Method.
\end{center}

The plot clearly shows self-similar fractal structure. The teardrop shapes are 
bordered by smaller, self-similar teardrop shapes. At several points, they seem
to converge onto a point. This is very interesting and a good representative
fractal structure.

\section*{Capacity Dimension}

\section*{Systems of ODEs}

\section*{Fractal Tree}

I implemented a program to draw the fractal tree described in the assignment and
found that 13 iterations were necessary to make the differences between 
successive segments indistinguishable. The results are shown in Figure 2.

\begin{center}
\includegraphics[scale=0.6]{fractaltree.png}
\\
Figure 2: Self-similar fractal tree with length ratio 0.6.
\end{center}

As the segment length ratio decreases to 0.5 and lower, the tree appears to thin
out. More space clears out between the branches. As the ratio increases past 
0.707, the branches begin to overlap. In these plots, the tree appears more like
a noisy grid. One interesting implication is that for values of the length ratio
greater that 0.707, many points inside the volume of the tree could be reached
by branching down two or more different paths. For smaller length ratios, each 
point in the volume of the tree corresponds uniquely to a branch path.



\end{document}