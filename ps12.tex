\documentclass[12pt, letterpaper]{article}
\title{Problem Set 12: Two-Body Problem}
\author{Ken Sheedlo}
\usepackage[pdftex]{graphicx}
\usepackage[margin=1.0in]{geometry}
\usepackage{verbatim}
\usepackage{amsfonts}
\usepackage{caption}
\usepackage{subcaption}

\begin{document}
\maketitle{}

To demonstrate a gravitational system with two bodies, I first wrote down the
differential equations for the two-body problem.

\begin{equation}
\dot{\vec{r_1}} = \vec{v_1}
\end{equation}
\begin{equation}
\dot{\vec{v_1}} = -Gm_2\frac{\vec{r_1}-\vec{r_2}}{|\vec{r_1}-\vec{r_2}|^3}
\end{equation}
\begin{equation}
\dot{\vec{r_2}} = \vec{v_2}
\end{equation}
\begin{equation}
\dot{\vec{v_2}} = -Gm_1\frac{\vec{r_2}-\vec{r_1}}{|\vec{r_2}-\vec{r_1}|^3}
\end{equation}

I expressed this as a single vector equation in my program code, and it was 
ready to run simulations inside of RK4. 

I plotted the simulation using initial values $\vec{r_1} = \vec{0}, \vec{v_1} = 
\vec{0}, \vec{r_2} = \left<1, 0, 0\right>$, and $\vec{v_2} = \left<0, 1, 
0\right>$. These are the values for the speed and velocity I found in problem set 
11. I used a gravitational constant $G=1.0$, and gave each star a mass of 0.5. 
By analysis, I found the period of one orbit to be $2\pi$, so I chose $\Delta t
=0.005$ and ran the simulation for 7600 timesteps. This worked out to be 
approximately 1256 timesteps per orbit period for just over 6 orbits. The results 
are shown in Figure 1.

\newpage

\begin{center}
\includegraphics[scale=0.6]{ps12.png}
\\
Figure 1: Simulated trajectory of two stars in orbit around one another.
\end{center}

The binary system does indeed orbit properly. I predicted that just over 6 orbits
would run, and the plot agrees with the prediction. The visual results are 
somewhat counterintuitive. Instead of an elliptical orbit, we see two cycloids 
offset. What is going on here is that the two stars are indeed orbiting around 
each other with eccentricity $e=0$. The center of mass of the system is moving in 
the $+y$ direction at a constant velocity. Conservation of momentum implies this, 
as star B is moving in the $+y$ direction while star A is at rest at $t=0$. 
Superimposing the motion of the center of mass on top of the motion of the stars 
around each other creates the cycloid. 

\end{document}