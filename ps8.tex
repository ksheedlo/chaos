\documentclass[12pt, letterpaper]{article}
\title{Problem Set 8: Delay Coordinate Embedding}
\author{Ken Sheedlo}
\usepackage[pdftex]{graphicx}
\usepackage{fullpage}
\usepackage{verbatim}
\usepackage{amsfonts}
\usepackage{caption}
\usepackage{subcaption}

\begin{document}
\maketitle{}

\section*{1. Constructing State-Space Trajectories}

I wrote a program to construct the state-space trajectory of the pendulum given 
a data set containing only $(\theta, t)$. To find $\omega$, I used an $O(h^2)$
three-point formula for numerical differentiation. This gave slightly better
results than the simple first-order forward difference formula. The results are
shown in Figure 1.

\begin{center}
\includegraphics[scale=0.6]{ps8_1.png}
\\
Figure 1: State-space trajectory constructed from data set 1.
\end{center}

To make the plot, I sampled at the period $\Delta t = 0.03$, which is every 
thirtieth data point. The shape of the plot is not a clean spiral, although it
is round and spiraling into the center. It is somewhat jagged, crossing over
itself in places and losing symmetry in the center. Two obvious factors could
be contributing to this behavior. First, because this data came from a real 
experiment, there could be errors in the precision of the observations. The data 
set gives 16 digits of precision for each $\theta$ value. It is doubtful that the 
instrument that was used to measure $\theta$ is really that precise. Second, 
numerical differentiation is problematic because of the way computers perform
arithmetic operations. Using a smaller step or a more precise formula forces the
machine to divide by a smaller number, increasing the potential for round-off
error. This could cause undesirable numerical damping effects in the plot.

\section*{2. Delay Coordinate Embedding}

To achieve delay-coordinate embedding, I wrote two Python routines, 
\texttt{embed.embed} and \texttt{plot.embedded}. I used \texttt{embed.embed} to
construct the list of $m$-vectors and \texttt{plot.embedded} to plot the $j^{th}$
and $k^{th}$ elements of each vector modulo $2\pi$. Keeping the routines for 
plotting and generating data separate made the program easier to test and debug.

\subsection*{a. Data Set 2}

Figure 2 shows a run of my delay-coordinate embedding program with $\tau=0.15$ s 
and $m=7$. 

\begin{center}
\includegraphics[scale=0.6]{ps8_2a.png}
\\
Figure 2: Delay-coordinate embedding over \texttt{data2}.
\end{center}

This is clearly a chaotic attractor. The trajectory scatters around the plot like 
a tangled rat's nest. It's not particularly clear where any of the trajectories 
will go given some elapsed time. It is interesting that the patterns near the 
corners and along the main diagonal are different from the patterns in between, 
for instance along $\theta(t+0.3) = \theta(t)\pm\pi$. It would be interesting to
explore further what this means in terms of the trajectory.

\subsection*{b. Data Set 3}

Figure 3 shows results of the delay-coordinate embedding program for various 
values of $\tau$. 

\begin{minipage}[t]{0.5\textwidth}
\begin{center}
\includegraphics[scale=0.55]{ps8_1_2b_cropped.png}
\includegraphics[scale=0.55]{ps8_4_2b_cropped.png}
\end{center}
\end{minipage}
\begin{minipage}[t]{0.5\textwidth}
\begin{center}
\includegraphics[scale=0.55]{ps8_3_2b_cropped.png}
\includegraphics[scale=0.55]{ps8_5_2b_cropped.png}
\end{center}
\end{minipage}
\begin{center}
Figure 3: Delay-coordinate embedding over \texttt{data3} with varying $\tau$.
\\
\end{center}

I think this means that the system is oscillating back and forth with these
parameters. So for $\tau=0.01$ the delay is too short for the trajectory to 
change much. For $\tau=0.5$ and $\tau=1.0$ the trajectory oscillates between 
two ranges of states, and for $\tau=1.5$ the system cycles back and forth each
time it is sampled, so it returns back to where it started.

\section*{3. Thought Experiments}
\subsection*{a. Takens' Theorem Requirements}

The driven pendulum is a nonautonomous system in $(\theta, \omega)$. So the 
actual dimension for a driven pendulum should be 3 $(\theta, \omega, t)$. Takens'
theorem states that the system can be embedded in $m$ dimensions if $m$ is 
greater than twice the box counting dimension of the attractor. The box counting
dimension can be no greater than 3, so $m=7$ is always sufficient to embed the
driven pendulum system. The undriven pendulum is an autonomous system in 
$(\theta, \omega)$. Its box counting dimension can be no greater than 2, so
$m=5$ would be sufficient for the undriven pendulum system.

\subsection*{b. Changing Dimension}

If I used $m=2$ the trajectory would not reflect the true dynamics of the 
system. If I used $m=25$ the reconstructed trajectory would not get to a better
picture, but noise might be amplified.

\subsection*{c. Changing Delay}

If I used $\tau=10^{-16}$ then there wouldn't be enough change and you would 
just see a diagonal line along $\theta(t+k\tau) = \theta(t)$. If I used 
$\tau=10^6$ then you might see dots scattered randomly, unless the natural 
period of the pendulum divided $10^6$ evenly in which case you would see a 
diagonal line because it would be sampling at the natural frequency.

\end{document}