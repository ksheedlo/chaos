\documentclass[12pt, letterpaper]{article}
\title{Problem Set 2: Bifurcation Diagrams}
\author{Ken Sheedlo}
\usepackage[pdftex]{graphicx}
\usepackage{fullpage}
\usepackage{verbatim}
\usepackage{amsfonts}
\usepackage{caption}
\usepackage{subcaption}

\begin{document}
\maketitle{}

\section*{Bifurcation Plot of the Logistic Map}

I wrote a program to draw the bifurcation plot of the logistic map for $2.8 < R
< 4$. I used $x_0 = 0.5$ and computed 1200 iterations of the map. I discarded 
the first 800 points from each run of the map. The interval between successive 
$R$ values is 0.001. Results are shown in Figure 1.

\begin{center}
\includegraphics[scale=0.6]{bifurc.png}
\\
Figure 1: Bifurcation plot of the logistic map.
\end{center}

\section*{Period Doubling Analysis}

I added a routine to my program to detect bifurcation points. I found 
bifurcation points at $R$-values 3.000, 3.448, 3.544, 3.564, and 3.568, as well 
as higher values in between chaotic sections. Using $b_1, b_2, ...,b_n$ to denote
bifurcation points on the same cascade, the Feigenbaum number $F$ can be 
approximated as

\begin{equation}
F = \frac{b_{k+1}-b_k}{b_{k+2}-b_{k+1}}
\end{equation}

where $k$ is an integer $1 \leq k \leq n$. Using Equation 1 and results from 
the program, I was able to estimate the Feigenbaum number experimentally. 
Results are shown in Table 2.

\begin{center}
\begin{tabular}{c | c | c | c | c}
$n$ & $b_n$ & $b_{n+1}$ & $b_{n+2}$ & Approx. $F$ \\ \hline
1 & 3.000 & 3.448 & 3.544 & 4.667 \\
2 & 3.448 & 3.544 & 3.564 & 4.800 \\
3 & 3.544 & 3.564 & 3.568 & 5.000 
\end{tabular}
\\
\vspace{1.0em}
Table 2: Estimating the Feigenbaum number experimentally.
\end{center}

The actual value of $F$ is 4.667. Note how the estimate actually diverges from the 
actual value as we take estimates further down the cascade. This is probably 
because as gaps between bifurcations get smaller and smaller, higher resolution is 
needed to sample them. Three digits is enough precision for the first estimate, but 
the other estimates could probably be improved by decreasing the interval between
successive $R$ values.

\section*{Period Doubling in the Henon Map}

Figure 3 shows a bifurcation plot of the Henon map with $b = 0.3$ and $0 < a < 
1.4$. 

\begin{center}
\includegraphics[scale=0.6]{henon_bifurc.png}
\\
Figure 3: Bifurcation plot of the Henon map.
\end{center}

I found a series of bifurcations at $a$-values 0.368, 0.909, 1.024, 1.051, and 
1.056. I also found a higher series at 1.226, 1.254, and 1.260, above a region 
of chaotic behavior. Table 4 shows estimates of $F$ produced from the data.

\begin{center}
\begin{tabular}{c | c | c | c | c}
$n$ & $b_n$ & $b_{n+1}$ & $b_{n+2}$ & Approx. $F$ \\ \hline
1 & 0.368 & 0.909 & 1.024 & 4.704 \\
2 & 0.909 & 1.024 & 1.051 & 4.259 \\
3 & 1.024 & 1.051 & 1.056 & 5.400 \\
6 & 1.226 & 1.254 & 1.260 & 4.667
\end{tabular}
\\
\vspace{1.0em}
Table 4: Estimating the Feigenbaum number using the Henon map.
\end{center}

The results from the Henon map are quite similar to those from the logistic map.
This is not surprising, as the Feigenbaum number holds for any 1D map with a 
quadratic maximum, including both the logistic and Henon maps. We should get
similar results. Of particular interest is the higher period doubling cascade 
which approximates the Feigenbaum number to three decimal places.

\end{document}