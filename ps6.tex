\documentclass[12pt, letterpaper]{article}
\title{Problem Set 6: Poincare Sections}
\author{Ken Sheedlo}
\usepackage[pdftex]{graphicx}
\usepackage{fullpage}
\usepackage{verbatim}
\usepackage{amsfonts}
\usepackage{caption}
\usepackage{subcaption}

\begin{document}
\maketitle{}

\section*{1. Simple Poincare Sections}

I wrote my simple Poincare section program and tested it on the pendulum from
PS4 problem 2(b). This is a pendulum swinging out a small arc near the origin.
The results are shown in Figure 1.

\begin{center}
\includegraphics[scale=0.6]{ps6_1a.png}
\\
Figure 1: Poincare section for a pendulum at its natural frequency.
\end{center}

All of the points on the section are clustered together near the starting point
at $[\theta, \omega] = [0.01, 0]$. This happens because the section is sampling
at the natural frequency of the pendulum and this particular pendulum has a 
periodic solution. It follows that the pendulum should be at about the same 
$[\theta, \omega]$ each time the section produces a sample.

Changing the sampling rate of the section such that it is not rationally related
causes all points on the trajectory to show up on the section. Figure 2 shows 
this result.

\begin{center}
\includegraphics[scale=0.6]{ps6_1b.png}
\\
Figure 2: Poincare section at an irrational multiple of the natural frequency.
\end{center}

Sampling at integer multiples of the natural frequency causes the trajectory to
always pierce the section in the same place. Sampling at half the natural 
frequency would pierce the section in two places, and sampling at one third the
natural frequency would pierce it in three places, and so on. Sampling at an
irrational multiple relative to the natural frequency causes the section to
eventually intersect the trajectory at every point on the trajectory. This is 
why Figure 1 shows points in one spot while Figure 2 shows points distributed 
over the entire trajectory.

Figure 3 shows a Poincare section of a chaotic pendulum trajectory. This section
was generated by sampling at the drive frequency over 1,000,000 points.

\begin{center}
\includegraphics[scale=0.6]{ps6_1c.png}
\\
Figure 3: Poincare section for a chaotic pendulum.
\end{center}

Increasing the step size by a factor of 4 while keeping the timespan covered by
the trajectory constant causes the plot to keep the same overall shape, but skew
out slightly. The results are shown in Figure 4.

\begin{center}
\includegraphics[scale=0.6]{ps6_1d.png}
\\
Figure 4: Poincare section for a chaotic pendulum with increased timestep.
\end{center}

Comparing Figures 3 and 4, one can see several clusters where trajectories tend
to pierce the section. The clusters are much tighter in Figure 3 than in Figure
4. This is because the timestep is 4 times smaller in Figure 3 than in Figure 4,
making Figure 3 a more accurate representation of the section. The simple 
Poincare section algorithm has $O(h)$ error where $h$ is the timestep, so Figure 
3 should be about 4 times more accurate than Figure 4.

\section*{2. Improved Poincare Sections}

A simple improvement to the algorithm involves taking a linear interpolation
between two points that cross the hyperplane to get a better approximation of 
the section. Figure 5 shows this technique applied to the same system shown in 
Figure 3.

\begin{center}
\includegraphics[scale=0.6]{ps6_2a.png}
\\
Figure 5: Improved Poincare section for a chaotic pendulum.
\end{center}

This plot clearly shows the improved accuracy with the better algorithm. Where
Figure 4 showed the section spreading out and becoming more sloppy, the section
tightens up nicely in Figure 5. 

Finally, I applied the improved algorithm to the chaotic trajectory with the
increased timestep. The results are shown in Figure 6.

\begin{center}
\includegraphics[scale=0.6]{ps6_2b.png}
\\
Figure 6: Chaotic pendulum with increased timestep and improved Poincare section.
\end{center}

This plot shows improved accuracy over Figure 4, but also shows artifacts across
the middle. These are probably a numerical glitch of some kind, since the 
trajectory itself is correct as shown in Figure 4.

\end{document}